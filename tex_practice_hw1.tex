\documentclass[a4paper, 11pt]{article}
\usepackage{kotex}
\usepackage{amsmath, amsfonts, amssymb, amsthm}
\usepackage{lineno}

\usepackage{color,graphicx}
\usepackage{hyperref}
\usepackage{epsfig,fullpage}
\usepackage{natbib,cite}
\usepackage{url}
\usepackage{subfigure}
\usepackage{soul}
\usepackage{enumitem}
\usepackage{booktabs}
\usepackage{caption}
\usepackage{array,multirow}
\usepackage{stackrel}
\usepackage{appendix}
\usepackage{geometry}

\newtheorem{theorem}{Theorem}
\newtheorem{assumption}{Assumption}
\newtheorem{corollary}{Corollary}
\newtheorem{lemma}{Lemma}
\newtheorem*{remark}{Remark}
\newtheorem{proposition}{Proposition}

\begin{document}
	\title{대학원 신입생 세미나: LaTex Homework}
	\author{권영욱 2020-26739}
	\maketitle
\subsection*{Theorem 1.6.8 (p.31, 5th edition)}	
Suppose $X_n\rightarrow X$ a.s. Let $g, h$ be a continuous functions with
\begin{itemize}
	\item[(i)] $g\ge 0$ and $g(x)\rightarrow \infty$ as $|x|\rightarrow \infty$,
	\item[(ii)] $|h(x)|/g(x)\rightarrow 0$ as $|x|\rightarrow \infty$,
\end{itemize}
and (iii) $Eh(X_n)\le K<\infty$ for all $n.$ \\
Then $Eh(X_n)\rightarrow Eh(X).$ \\
\\
\textit{Proof.}	By subtracting a constan from $h,$ we can suppose without loss of generality that $h(0)=0.$ Pick M large so that $P(|X|=M)=0$ and $g(x)>0$ when $|x|\ge M.$ Given a random variable $Y,$ let $\bar{Y}=Y1_{(|Y|\le M)}.$ Since $P(|X|=M)=0,$ $\bar{X}_{n}\rightarrow \bar{X}$ a.s. Since $h(\bar{X}_{n})$ is bounded and h is continuous, it follow from the bounded convergence theorem that 
\begin{equation}
Eh(\bar{X}_{n})\rightarrow Eh(\bar{X})
\end{equation}
To control the effect of the truncation, we use the following:
\begin{equation}
|Eh(\bar{Y})-Eh(Y)|\le E|h(\bar{Y})-h(Y)|\le E(|h(Y)|;|Y|\ge M) \le \epsilon_{M}Eg(Y)
\end{equation}
where $\epsilon_{M}=sup\{|h(x)|/g(x): |x|\ge M\}.$ To check the second inequality, note that when $|Y|\le M,$ $\bar{Y}=Y,$ and we have supposed $h(0)=0.$ The third inequality follows form the definition of $\epsilon_{M}.$ \\
Taking $Y=X_{n}$ in (2) and using (iii), it follows that 
\begin{equation}
|Eh(\bar{X}_{n})-Eh(X_n)|\le K\epsilon_{M}
\end{equation}
To estimate $|Eh(\bar{X})-Eh(X)|,$ we observe that $g\ge 0$ and g is contiuous, so Fatou's lemma implies
\begin{equation*}
Eg(X)\le \lim_{n}inf Eg(X_n)\le K
\end{equation*}
Taking $Y=X$ in (2) gives 
\begin{equation}
|Eh(\bar{X})-Eh(X)|\le K\epsilon_{M}
\end{equation}
The triangle inequality implies 
\begin{eqnarray*}
|Eh(X_n)-Eh(X)| & \le & |Eh(X_n)-Eh(\bar{X}_{n})| \\
& + & |Eh(\bar{X}_{n})-Eh(\bar{X})| + |Eh(\bar{X})-Eh(X)| \\
\end{eqnarray*}
Taking limits and using (1), (3), (4), we have 
\begin{equation*}
\lim_{n}sup|Eh(X_n)-Eh(X)|\le 2K\epsilon_{M}
\end{equation*}
which proves the desired result since $K<\infty$ and $\epsilon_{M} \rightarrow 0$ as $M\rightarrow \infty.$
\end{document}